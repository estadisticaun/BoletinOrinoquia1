% Options for packages loaded elsewhere
\PassOptionsToPackage{unicode}{hyperref}
\PassOptionsToPackage{hyphens}{url}
%
\documentclass[
]{book}
\usepackage{lmodern}
\usepackage{amssymb,amsmath}
\usepackage{ifxetex,ifluatex}
\ifnum 0\ifxetex 1\fi\ifluatex 1\fi=0 % if pdftex
  \usepackage[T1]{fontenc}
  \usepackage[utf8]{inputenc}
  \usepackage{textcomp} % provide euro and other symbols
\else % if luatex or xetex
  \usepackage{unicode-math}
  \defaultfontfeatures{Scale=MatchLowercase}
  \defaultfontfeatures[\rmfamily]{Ligatures=TeX,Scale=1}
\fi
% Use upquote if available, for straight quotes in verbatim environments
\IfFileExists{upquote.sty}{\usepackage{upquote}}{}
\IfFileExists{microtype.sty}{% use microtype if available
  \usepackage[]{microtype}
  \UseMicrotypeSet[protrusion]{basicmath} % disable protrusion for tt fonts
}{}
\makeatletter
\@ifundefined{KOMAClassName}{% if non-KOMA class
  \IfFileExists{parskip.sty}{%
    \usepackage{parskip}
  }{% else
    \setlength{\parindent}{0pt}
    \setlength{\parskip}{6pt plus 2pt minus 1pt}}
}{% if KOMA class
  \KOMAoptions{parskip=half}}
\makeatother
\usepackage{xcolor}
\IfFileExists{xurl.sty}{\usepackage{xurl}}{} % add URL line breaks if available
\IfFileExists{bookmark.sty}{\usepackage{bookmark}}{\usepackage{hyperref}}
\hypersetup{
  pdftitle={Boletín Estadístico Sede Orinoquía},
  pdfauthor={ Universidad Nacional de Colombia   Sede Orinoquía},
  hidelinks,
  pdfcreator={LaTeX via pandoc}}
\urlstyle{same} % disable monospaced font for URLs
\usepackage{longtable,booktabs}
% Correct order of tables after \paragraph or \subparagraph
\usepackage{etoolbox}
\makeatletter
\patchcmd\longtable{\par}{\if@noskipsec\mbox{}\fi\par}{}{}
\makeatother
% Allow footnotes in longtable head/foot
\IfFileExists{footnotehyper.sty}{\usepackage{footnotehyper}}{\usepackage{footnote}}
\makesavenoteenv{longtable}
\usepackage{graphicx}
\makeatletter
\def\maxwidth{\ifdim\Gin@nat@width>\linewidth\linewidth\else\Gin@nat@width\fi}
\def\maxheight{\ifdim\Gin@nat@height>\textheight\textheight\else\Gin@nat@height\fi}
\makeatother
% Scale images if necessary, so that they will not overflow the page
% margins by default, and it is still possible to overwrite the defaults
% using explicit options in \includegraphics[width, height, ...]{}
\setkeys{Gin}{width=\maxwidth,height=\maxheight,keepaspectratio}
% Set default figure placement to htbp
\makeatletter
\def\fps@figure{htbp}
\makeatother
\setlength{\emergencystretch}{3em} % prevent overfull lines
\providecommand{\tightlist}{%
  \setlength{\itemsep}{0pt}\setlength{\parskip}{0pt}}
\setcounter{secnumdepth}{5}
\usepackage{booktabs}
\usepackage[left=3cm,right=3cm,top=6cm,bottom=6cm]{geometry}
\usepackage[]{natbib}
\bibliographystyle{apalike}

\title{Boletín Estadístico Sede Orinoquía}
\author{ Universidad Nacional de Colombia Sede Orinoquía}
\date{Última actualización: 22 de enero de 2021}

\begin{document}
\maketitle

{
\setcounter{tocdepth}{1}
\tableofcontents
}
\hypertarget{portada}{%
\chapter*{Portada}\label{portada}}
\addcontentsline{toc}{chapter}{Portada}

\textbf{Comentarios, sugerencias e inquietudes}

\textbf{CLAUDIA PATRICIA JOYA JOYA}
Asesora Dirección
Email: \href{mailto:cpjoyaj@unal.edu.co}{\nolinkurl{cpjoyaj@unal.edu.co}}
Teléfono: (1) 3165000 extensión 34809

\textbf{GEOVANNY MANZANO SANCHEZ}
Profesional Sistema Integrado de Gestión Académica Administrativa y Ambiental - SIGA
Email: \href{mailto:gmanzanos@unal.edu.co}{\nolinkurl{gmanzanos@unal.edu.co}}
Teléfono: (1) 3165000 extensión 29721

\textbf{Sistema Estadístico UNAL}

El Boletín Estadísco de la Sede Amazonía de la Universidad Nacional de Colombia hace parte integral del \emph{sistema estadístico institucional}. Para aquellos interesados en conocer la información estadística general de la Universidad así como de las demás sedes que hacen parte de esta institución, los invitamos a ingresar al sitio web de \href{http://estadisticas.unal.edu.co/home/}{estadísticas institucionales} y a conocer los porqués de este sitio expresados en el siguiente video institucional.

\hypertarget{Presenta}{%
\chapter*{Presentación}\label{Presenta}}
\addcontentsline{toc}{chapter}{Presentación}

Bienvenidos a la primera edición en línea e interactiva del Boletín Estadístico de la Universidad Nacional de Colombia \href{http://orinoquia.unal.edu.co/}{Sede Orinoquia}, un sitio web con la información y las cifras más representativas del desarrollo de la función misional representada en los procesos de formación, investigación y extensión desde el 2008 a la fecha, así como estadísticas resultantes del proceso de Gestión y Fomento Socioeconómico en el marco de las acciones de Bienestar Universitario.

Con esta boletín estadístico se pretende cumplir con dos objetivos; el primero orientado a mantener registros actualizados y periódicos de algunas de las variables más importantes de la función misional de la sede como base de consulta para docentes, funcionarios, partes interesadas y comunidad en general. El segundo objetivo pretende facilitar la toma de decisiones que permita concretar la visión estratégica de la Institución de una forma coherente, metódica y eficaz, teniendo en cuenta que se hace necesaria la medición de la gestión y el desempeño como parte de los procesos de autoevaluación institucional.

Agradezco a todas los procesos que aportaron la información para la construcción de este boletín. Los comentarios o sugerencias relacionados con esta publicación pueden ser enviados a los correos electrónicos \textbf{\href{mailto:siga_ori@unal.edu.co}{\nolinkurl{siga\_ori@unal.edu.co}}} o \textbf{\href{mailto:planeación_ori@unal.edu.co}{\nolinkurl{planeación\_ori@unal.edu.co}}}

Cordialmente,

\textbf{OSCAR EDUARDO SUAREZ MORENO}
Director Sede Orinoquia

\hypertarget{intro}{%
\chapter*{Introducción}\label{intro}}
\addcontentsline{toc}{chapter}{Introducción}

La Universidad Nacional de Colombia Sede Orinoquia, en un esfuerzo por consolidar y divulgar la información referente a su quehacer misional en la región Orinoquia, presenta a la comunidad universitaria y a la sociedad en general una nueva edición online del Boletín Estadístico de la Sede Orinoquía.

En la presente publicación - ``Boletín Estadístico de la Sede Orinoquía-, se muestra el desempeño de la Sede desde su creación hasta la actualidad. El documento hace un amplio recorrido a la historia de la Sede, representado en estadísticas basadas en la documentación física existente desde el inicio de las actividades de la Sede.

En el cumplimento de cada una de sus funciones, la sede ha tenido logros y avances importantes para la Universidad y la Región de la Orinoquia, los cuales se dan a conocer a la comunidad universitaria y a la opinión pública en general mediante esta publicación.

La Sede Orinoquia ha hecho un esfuerzo de sistematización de la información para establecer las estadísticas y los indicadores de tal forma que se haga visible la gestión de la Sede, en cumplimiento de la Misión de la Universidad Nacional de Colombia en la región de la Orinoquia.

Los datos representados en cifras estadísticas proporcionan información veraz acerca de los procesos misionales: formación, investigación y extensión. Además, presenta información sobre egresados, personal docente y administrativo, presupuestos, bienestar universitario e infraestructura.

Se pretende con esta publicación mantener registros actualizados y periódicos de algunas de las variables más importantes de la Sede, como una herramienta base de consulta que permite orientar la toma de decisiones y así concretar la
visión estratégica de la Institución de una forma coherente, metódica y eficaz, teniendo en cuenta que se hace necesaria la medición de la gestión y el desempeño como parte de la autoevaluación institucional para la rendición de cuentas a la comunidad universitaria, entidades externas y/o grupos de interés.

\hypertarget{mision}{%
\section*{Misión}\label{mision}}
\addcontentsline{toc}{section}{Misión}

La Universidad Nacional de Colombia, como Universidad de la nación, fomenta el acceso con equidad al sistema educativo colombiano, provee la mayor oferta de programas académicos, forma profesionales competentes y socialmente responsables. Contribuye a la elaboración y resignificación del proyecto de nación, estudia y enriquece el patrimonio cultural, natural y ambiental del país. Como tal lo asesora en los órdenes científico, tecnológico, cultural y artístico con autonomía académica e investigativa.

\hypertarget{vision}{%
\section*{Visión 2030}\label{vision}}
\addcontentsline{toc}{section}{Visión 2030}

En el año 2030 somos la principal universidad colombiana, reconocida por su contribución a la
nación, y por su excelencia en los procesos de formación, investigación, e innovación social y
tecnológica. Nuestra capacidad de reinventarnos nos ha llevado a tener una organización académica y administrativa novedosa, flexible, eficiente y sostenible, con comunicación transparente y efectiva en su interior, con la nación y con el mundo, y comprometida con los procesos
de transformación social requeridos para alcanzar una sociedad equitativa, incluyente y en paz.

\hypertarget{hist}{%
\section*{Historia}\label{hist}}
\addcontentsline{toc}{section}{Historia}

La Sede Orinoquía de la Universidad Nacional de Colombia fue creada gracias al Acuerdo 40 de 1996 del Consejo Superior Universitario. En el año 1997, la sede entró en funcionamiento en la ciudad de Arauca y, así mismo, se creó el Instituto de Estudios de la Orinoquía. Más tarde, en el año 2005, a través del Acuerdo 011 de 2005, la sede se transformó en Sede de Presencia Nacional de Orinoquía; de ese modo, la Universidad Nacional de Colombia reconoce y garantiza su presencia en los Llanos Orientales, una zona estratégica tanto para el desarrollo económico, político y social como para el acervo cultural del país, donde consolida y demuestra su trayectoría académica e investigativa.

Es así como desde su creación, la Sede de Presencia Nacional de Orinoquía ha desarrollado tres tipos de formación en programas de pregrado. En primer lugar, con cohortes de programas completos: Enfermería, Ingeniería Ambiental e Ingeniería Agronómica. En segundo lugar, con un Programa Especial de Admisión de Ingreso por Áreas, que consistió en la realización de un semestre de fundamentación y posteriormente la admisión a un área del conocimiento ya sea ciencias o ingeniería. En tercer lugar, con un Programa Especial de Admisión y Movilidad académica -- Peama, el cual permite la admisión semestral a la Universidad Nacional de Colombia de jóvenes residentes en Arauca, Casanare, Guainía, Guaviare y Vichada a 73 programas curriculares de las sedes Bogotá, Medellín, Manizales y Palmira.

La oferta de programas de posgrado ha estado orientada a la demanda de profesionales y ,además, estos programas se han ofertado por medio de convenio con otras sedes de la Universidad Nacional de Colombia.

De igual forma, la investigación se ha desarrollado mediante diferentes modalidades, permitiendo desde la articulación de esta con los programas de pregrado y posgrado desarrollados por la Sede en convenio de cooperación, hasta el acceso a recursos de cofinanciación y las iniciativas de docentes vinculados a la Sede.

Por un lado, a partir del 2009, la investigación se fortaleció mediante la apertura de convocatorias de investigación, que permitieron, primero, la ejecución de proyectos de investigación a través de cofinanciación; segundo, la movilidad de estudiantes y profesores con fines investigativos; tercero, la vinculación de grupos de investigación de otras sedes de la Universidad para el desarrollo de proyectos en temas de la Orinoquia; cuarto, la creación y fortalecimiento de semilleros de investigación y, finalmente, el desarrollo de actividades encaminadas a promover la divulgación de los resultados de investigación obtenidos en la comunidad llanera, académica y científica a nivel regional y nacional.

La extensión en la Sede se ha desplegado con los servicios académicos, la educación continua, los servicios de educación, la extensión solidaria, las prácticas y las pasantías. Aunque cabe anotar que, durante los últimos años, se ha hecho especial énfasis en los servicios de educación gracias al laboratorio de la Sede y a la educación continua con la realización de congresos, seminarios, talleres y conferencias.

Así, la investigación y la extensión en la sede se expanden cada vez más gracias a proyectos como la Granja Experimental El Cairo, proyectada como un modelo agropecuario autosostenible para la región, que sirve de soporte y complemento al Laboratorio de Suelos, Aguas y Foliares, que es único en el departamento y el segundo en la Orinoquía, proyectado también como herramienta de tenificación al sector productivo.

Por otro lado, El apoyo a la misión institucional se ha fundamentado en el fortalecimiento de la infraestructura física y tecnológica, contando con canales de comunicaciones para servicio de videoconferencia, voz sobre IP, canal
WAN, acceso a sistemas de información propios, dos salas de sistemas, aulas de clase, biblioteca con acceso a todos los servicios del Sistema Nacional de Bibliotecas (Sinab), laboratorios para la docencia y espacios de bienestar.

\hypertarget{ubica}{%
\section*{Ubicación}\label{ubica}}
\addcontentsline{toc}{section}{Ubicación}

La Sede Orinoquia de la Universidad Nacional de Colombia se localiza a nueve kilómetros de la zona urbana del municipio de Arauca, sobre el costado izquierdo de la vía que conduce al Complejo Petrolero de Caño Limón, en el departamento de Arauca.

\hypertarget{peama}{%
\section*{PEAMA}\label{peama}}
\addcontentsline{toc}{section}{PEAMA}

\textbf{Programa especial de admisión y movilidad académica -- Peama}

Los procesos de formación en pregrado en la Universidad Nacional de Colombia Sede Orinoquía se desarrollan a través del programa especial de admisión y movilidad académica -- Peama. Este programa fue creado por la Universidad a través del Acuerdo del Consejo Superior Universitario N.° 025 de 2007, con el objetivo de participar activamente en el desarrollo social de las regiones fronterizas, a través de la formación profesional de los futuros líderes científicos, empresariales y políticos del país, siendo el modelo en el que opera la formación en las sedes de presencia nacional siendo estas la Sede Amazonia, Caribe, Tumaco y Orinoquía.

Aspectos como la escasa oferta de programas de pregrado en las regiones ubicadas en las fronteras del país, la dificultad económica para llevar carreras completas a estas regiones, una posible saturación del mercado laboral por la reducida oferta de programas de pregrado, el desarrollo regional que supone la presencia de diferentes áreas del conocimiento teniendo en cuenta las vocaciones de las regiones y las fortalezas institucionales de la Universidad Nacional de Colombia, fueron las situaciones que originaron la creación del Peama, para las Sedes de Presencia Nacional. El programa consiste en la oferta de programas curriculares de pregrado de las sedes Bogotá, Manizales, Medellín y Palmira a bachilleres residentes en las regiones de influencia de las Sedes de Presencia Nacional. Para el caso de la Sede Orinoquía, participan los departamentos de Arauca, Casanare, Guainía, Guaviare y Vichada. El Peama es especial en la forma de admitir a los estudiantes en la movilidad al inicio de sus estudios y en el apoyo metodológico. Se utiliza el sistema de telepresencia como instrumento educativo.

\textbf{Etapas}

El Peama tiene las siguientes etapas de formación:

\emph{\textbf{Etapa inicial}}: una vez admitido, el estudiante inicia estudios en la Sede de Presencia Nacional, en donde cursará algunas asignaturas. Esta etapa inicial podrá variar para cada estudiante según su desempeño en el examen de admisión, los requerimientos del programa curricular al que haya sido admitido y de acuerdo con la disponibilidad de los programas que se puedan ofrecer en la Sede de Presencia Nacional.

\emph{\textbf{Etapa de movilidad}}: iniciará movilidad a las sedes donde se ofrece el programa seleccionado. Como parte de la movilidad académica, el estudiante continuará los cursos del plan de estudios establecido, en la sede que ofrece el programa en el cual fue admitido.

\emph{\textbf{Etapa final}}: Para terminar el programa, el estudiante se desplazará a la Sede de Presencia Nacional con el fin de realizar su trabajo de grado. Cuando esto no sea posible, el estudiante deberá hacer su trabajo de grado, preferentemente, en temas de interés para su región.

\hypertarget{Prog}{%
\chapter{Programas Académicos}\label{Prog}}

\hypertarget{pregrado}{%
\section{Pregrado}\label{pregrado}}

\hypertarget{postgrado}{%
\section{Postgrado}\label{postgrado}}

\hypertarget{Aspirantes}{%
\chapter{Aspirantes y Admitidos}\label{Aspirantes}}

Este capítulo presenta el consolidado de las principales características asociadas a la información estadística oficial de los aspirantes y admitidos a pregrado y postgrado en la sede Orinoquía de la Universidad Nacional de Colombia. A continuación, se presenta una breve descripción de las secciones que hacen parte de este capítulo así como la ubicación del sitio web en donde se presentan las definiciones, los estándares y las codificaciones/clasificaciones que hacen parte de la información acá contenida (\emph{metadatos}) y cuya exploración y lectura, sin duda, facilitará el entendimiento de las cifras asociadas a las poblaciones de aspirantes y admitidos en esta sede de la Universidad.

\textbf{Secciones}

El proceso de formación en pregrado en la Sede Orinoquía se da a la luz de los lineamientos definidos en el Programa Especial de Admisión y Movilidad Académica \protect\hyperlink{peama}{PEAMA}; en contraste, la formación postgradual en en esta sede, se rige bajo los parámetros nacionales que orientan este nivel de formación.

\begin{itemize}
\item
  \protect\hyperlink{AspPre}{Aspirantes a pregrado}: contiene la información oficial del total de aspirantes a pregrado en la Sede Orinoquía de la Universidad Nacional de Colombia y que se incribieron a través del Programa Especial de Admisión y Movilidad Académica \protect\hyperlink{peama}{PEAMA}.
\item
  \protect\hyperlink{AdmPre}{Admitidos a pregrado}: contiene la información oficial del total de admitidos a pregrado en la Sede Orinoquía de la Universidad Nacional de Colombia y que se incribieron a través del Programa Especial de Admisión y Movilidad Académica \protect\hyperlink{peama}{PEAMA}.
\item
  \protect\hyperlink{AspPos}{Aspirantes a postgrado}: contiene la información oficial del total de aspirantes a los programas de postgrado adscritos a la Sede Orinoquía de la Universidad Nacional de Colombia.
\item
  \protect\hyperlink{AdmPos}{Admitidos a postgrado}: contiene la información oficial del total de aspirantes admitidos a los programas de postgrado adscritos a la Sede Orinoquía de la Universidad Nacional de Colombia.
\end{itemize}

\textbf{Metadatos}

La construcción de las cifras oficiales de aspirantes y admitidos a la Sede Orinoquía, las definiciones que hacen parte de estas así como las codificaciones y clasificaciones aquí empleadas se encuentran contenidas en la sección \textbf{Aspirantes y Admitidos} del capítulo de \emph{Metadatos} de las cifras oficiales generales que hacen parte de la página de \href{http://estadisticas.unal.edu.co/home/}{estadísticas} de la Universidad Nacional de Colombia. Invitamos a los lectores a explorar y conocer estos metadatos los cuales, además de orientar y facilitar el entendimiento de la información acá expuesta, se encuentran disponibles en el siguiente enlace.

\begin{itemize}
\tightlist
\item
  \href{http://estadisticas.unal.edu.co/menu-principal/cifras-generales/metadatos/cifras-generales/}{Metadatos Cifras Oficiales Universidad Nacional de Colombia}
\end{itemize}

\hypertarget{AspPre}{%
\section{Aspirantes a pregrado}\label{AspPre}}

A continuación, se presentan las principales características asociadas a los aspirantes a pregrado de la \textbf{Sede Orinoquía} de la Universidad Nacional de Colombia. En específico, se presenta la evolución histórica de los aspirantes a pregrado desde diferentes perspectivas: general, sexo, grupos de edad, estrato socioeconómico, tasa de absorción y departamentos y municipios de residencia de los mismos. Para cada una de las variables analizadas se presenta la evolución histórica (\emph{serie de tiempo}) así como el comportamiento actual (\emph{estado actual}) derivado de las últimas mediciones disponibles.

\hypertarget{evoluciuxf3n-histuxf3rica}{%
\subsection{Evolución Histórica}\label{evoluciuxf3n-histuxf3rica}}

A continuación, la Figura \ref{fig:F1AspPre}, presenta la evolución histórica -\emph{desde el periodo 20082}-, del total de aspirantes a pregrado en la sede.

\includegraphics{Boletin-Orinoquia_files/figure-latex/F1AspPre-1.pdf}

\hypertarget{informaciuxf3n-por-sexo}{%
\subsection{Información por sexo}\label{informaciuxf3n-por-sexo}}

A continuación, las figuras \ref{fig:F2AspPre} y \ref{fig:F3AspPre} presentan, respectivamente, la evolución histórica y el comportamiento actual del total de aspirantes a pregrado según el sexo biológico.

\includegraphics{Boletin-Orinoquia_files/figure-latex/F2AspPre-1.pdf}
\includegraphics{Boletin-Orinoquia_files/figure-latex/F3AspPre-1.pdf}

\hypertarget{informaciuxf3n-por-edad}{%
\subsection{Información por edad}\label{informaciuxf3n-por-edad}}

A continuación, las figuras \ref{fig:F4AspPre} y \ref{fig:F5AspPre} presentan, respectivamente, la evolución histórica y el comportamiento actual del total de aspirantes a pregrado por grupos de edad.

\includegraphics{Boletin-Orinoquia_files/figure-latex/F4AspPre-1.pdf}
\includegraphics{Boletin-Orinoquia_files/figure-latex/F5AspPre-1.pdf}

\hypertarget{informaciuxf3n-por-estrato-socioeconuxf3mico}{%
\subsection{Información por estrato socioeconómico}\label{informaciuxf3n-por-estrato-socioeconuxf3mico}}

A continuación, las figuras \ref{fig:F6AspPre} y \ref{fig:F7AspPre} presentan, respectivamente, la evolución histórica y el comportamiento actual del total de aspirantes a pregrado según el estrato socioeconómico de las viviendas en donde estos residen.

\includegraphics{Boletin-Orinoquia_files/figure-latex/F6AspPre-1.pdf}
\includegraphics{Boletin-Orinoquia_files/figure-latex/F7AspPre-1.pdf}

\hypertarget{tasa-de-absorciuxf3n}{%
\subsection{Tasa de absorción}\label{tasa-de-absorciuxf3n}}

A continuación, las figuras \ref{fig:F8AspPre} y \ref{fig:F9AspPre} presentan, respectivamente, la evolución histórica y el comportamiento actual del total de aspirantes admitidos a pregrado en la Sede Orinoquía de la Universidad Nacional de Colombia.

\includegraphics{Boletin-Orinoquia_files/figure-latex/F8AspPre-1.pdf}
\includegraphics{Boletin-Orinoquia_files/figure-latex/F9AspPre-1.pdf}

\hypertarget{tablas-cifras-agregadas}{%
\subsection{Tablas cifras agregadas}\label{tablas-cifras-agregadas}}

A continuación se presentan, a través de tablas, los agregados/consolidados históricos del total de aspirantes a pregrado de la Sede Orinoquía por departamentos y municipios de procedencia.

Los interesados en \emph{imprimir}, \emph{copiar} o \emph{descargar} estas cifras, pueden hacerlo a través de las múltiples opciones que se ofrecen en la parte superior izquierda de cada una de las tablas que se presentan a continuación (Copiar, CSV, Excel, PDF e Imprimir). Así mismo estas tablas, dada su naturaleza web, permiten filtrar los resultados por aquellas variables de interés.

\hypertarget{departamentos-y-municipios-de-procedencia}{%
\subsubsection{Departamentos y municipios de procedencia}\label{departamentos-y-municipios-de-procedencia}}

A continuación, la tabla \ref{fig:F10AspPre} presenta el acumulado \textbf{histórico}, por \textbf{años y semestres}, de los aspirantes a pregrado por departamentos y municipios de procedencia en la Sede Orinoquía.

\begin{figure}
\centering
\includegraphics{Boletin-Orinoquia_files/figure-latex/F10AspPre-1.pdf}
\caption{\label{fig:F10AspPre}Fuente: Dirección Nacional de Planeación y Estadística con base en información de la Dirección Nacional de Admisiones}
\end{figure}

\hypertarget{AdmPre}{%
\section{Admitidos a pregrado}\label{AdmPre}}

A continuación, se presentan las principales características asociadas a los admitidos a pregrado de la Sede Orinoquía de la Universidad Nacional de Colombia. En específico, se presenta la evolución histórica de los admitidos a pregrado desde diferentes perspectivas: general, sexo, grupos de edad, estrato socioeconómico, departamentos y municipios de residencia así como los programas académicos, las facultades y las sedes andinas en las que estos se encuentran ubicados. Para cada una de las variables analizadas se presenta la evolución histórica (\emph{serie de tiempo}) así como el comportamiento actual (\emph{estado actual}) derivado de las últimas mediciones disponibles.

\hypertarget{evoluciuxf3n-histuxf3rica-1}{%
\subsection{Evolución Histórica}\label{evoluciuxf3n-histuxf3rica-1}}

A continuación, la Figura \ref{fig:F1AdmPre}, presenta la evolución histórica -\emph{desde el periodo 20082}-, del total de admitidos a pregrado en la sede.

\includegraphics{Boletin-Orinoquia_files/figure-latex/F1AdmPre-1.pdf}

\hypertarget{informaciuxf3n-por-sexo-1}{%
\subsection{Información por sexo}\label{informaciuxf3n-por-sexo-1}}

A continuación, las figuras \ref{fig:F2AdmPre} y \ref{fig:F3AdmPre} presentan, respectivamente, la evolución histórica y el comportamiento actual del total de admitidos a pregrado según el sexo biológico.

\includegraphics{Boletin-Orinoquia_files/figure-latex/F2AdmPre-1.pdf}
\includegraphics{Boletin-Orinoquia_files/figure-latex/F3AdmPre-1.pdf}

\hypertarget{informaciuxf3n-por-edad-1}{%
\subsection{Información por edad}\label{informaciuxf3n-por-edad-1}}

A continuación, las figuras \ref{fig:F4AdmPre} y \ref{fig:F5AdmPre} presentan, respectivamente, la evolución histórica y el comportamiento actual del total de admitidos a pregrado por grupos de edad.

\includegraphics{Boletin-Orinoquia_files/figure-latex/F4AdmPre-1.pdf}
\includegraphics{Boletin-Orinoquia_files/figure-latex/F5AdmPre-1.pdf}

\hypertarget{informaciuxf3n-por-estrato-socioeconuxf3mico-1}{%
\subsection{Información por estrato socioeconómico}\label{informaciuxf3n-por-estrato-socioeconuxf3mico-1}}

A continuación, las figuras \ref{fig:F6AdmPre} y \ref{fig:F7AdmPre} presentan, respectivamente, la evolución histórica y el comportamiento actual del total de admitidos a pregrado según el estrato socioeconómico de las viviendas en donde estos residen.

\includegraphics{Boletin-Orinoquia_files/figure-latex/F6AdmPre-1.pdf}
\includegraphics{Boletin-Orinoquia_files/figure-latex/F7AdmPre-1.pdf}

\hypertarget{tablas-cifras-agregadas-1}{%
\subsection{Tablas cifras agregadas}\label{tablas-cifras-agregadas-1}}

A continuación se presentan, a través de tablas, los agregados/consolidados históricos del total de admitidos a pregrado de la Sede Orinoquía por departamentos y municipios de procedencia, así como los programas académicos y las sedes andinas a las que estos pertecenecen.

Los interesados en \emph{imprimir}, \emph{copiar} o \emph{descargar} estas cifras, pueden hacerlo a través de las múltiples opciones que se ofrecen en la parte superior izquierda de cada una de las tablas que se presentan a continuación (Copiar, CSV, Excel, PDF e Imprimir). Así mismo estas tablas, dada su naturaleza web, permiten filtrar los resultados por aquellas variables de interés.

\hypertarget{departamentos-y-municipios-de-procedencia.}{%
\subsubsection{Departamentos y municipios de procedencia.}\label{departamentos-y-municipios-de-procedencia.}}

A continuación, la tabla \ref{fig:F8AdmPre} presenta el acumulado \textbf{histórico} por \textbf{años y semestres} en la Sede Orinoquía de los admitidos a pregrado por departamentos y municipios de procedencia.

\includegraphics{Boletin-Orinoquia_files/figure-latex/F8AdmPre-1.pdf}

\hypertarget{programas-acaduxe9micos-de-pregrado}{%
\subsubsection{Programas académicos de pregrado}\label{programas-acaduxe9micos-de-pregrado}}

A continuación, la tabla \ref{fig:F9AdmPre} presenta el acumulado \textbf{histórico} por \textbf{años y semestres} en la Sede Orinoquía de los admitidos a pregrado por programas académicos y sedes andinas en las que estos se encuentran adscritos.

\begin{figure}
\centering
\includegraphics{Boletin-Orinoquia_files/figure-latex/F9AdmPre-1.pdf}
\caption{\label{fig:F9AdmPre}Fuente: Dirección Nacional de Planeación y Estadística con base en información de la Dirección Nacional de Admisiones}
\end{figure}

\hypertarget{Estudiantes}{%
\chapter{Estudiantes}\label{Estudiantes}}

Este capítulo presenta el consolidado de las principales características asociadas a la información estadística oficial de los estudiantes matriculados en pregrado y postgrado en la sede Orinoquía de la Universidad Nacional de Colombia. A continuación, se presenta una breve descripción de las secciones que hacen parte de este capítulo así como la ubicación del sitio web en donde se presentan las definiciones, los estándares y las codificaciones/clasificaciones que hacen parte de la información acá contenida (\emph{metadatos}) y cuya exploración y lectura, sin duda, facilitará el entendimiento de las cifras asociadas a las poblaciones de estudiantes matriculados.

\textbf{Secciones}

El proceso de formación en pregrado en la Sede Orinoquía se da a la luz de los lineamientos definidos en el Programa Especial de Admisión y Movilidad Académica el cual, como se presenta en la sección \protect\hyperlink{peama}{PEAMA} del presente boletín, consta de tres etapas: \emph{etapa inicial}, \emph{etapa de movilidad} y \emph{etapa final}. Teniendo en cuenta lo anterior, en este capítulo se presentan las estadísticas oficiales asociadas a cuatro poblaciones de estudiantes matriculados en la Sede Orinoquía de la Universidad Nacional de Colombia.

\begin{itemize}
\item
  \protect\hyperlink{MatPre}{Matriculados en pregrado}: contiene la información oficial del total de estudiantes matricualdos en pregrado en la Universidad y que han sido admitidos a través de la Sede Orinoquía. Incluye los estudiantes de pregrado que se encuentran en las tres etapas de formación: \emph{inicial}, \emph{movilidad} y \emph{final}.
\item
  \protect\hyperlink{MatPreIni}{Matriculados en pregrado etapa inicial}: contiene la información oficial del total de estudiantes matriculados en pregrado en la Universidad, que han sido admitidos a través de la Sede Orinoquía y que encuentran en la etapa inicial de su proceso de formación académica. Es decir, hace referencia a los estudiantes de pregrado que se encuentran ubicados de manera fisíca en el campus de la Sede Orinoquía.
\item
  \protect\hyperlink{MatPreMov}{Matriculados en pregrado etapa movilidad}: contiene la información oficial del total de estudiantes matriculados en pregrado en la Universidad, que han sido admitidos a través de la Sede Orinoquía y que encuentran en la etapa de movilidad en su proceso de formación académica. Es decir, hace referencia a los estudiantes de pregrado, admitidos a través de la Sede Orinoquía y que se encuentran cursando sus estudios en los campus de las sedes Bogotá, Medellín, Manizales o Palmira.
\item
  \protect\hyperlink{MatPos}{Matriculados en postgrado}: Contiene la información de los estudiantes matriculados en los programas de postgrado adscritos a la Sede Orinoquía de la Universidad Nacional de Colombia. No incluye la información de estudiantes de postgrado ubicados en la sede que se encuentren o hayan cursado, mediante la modalidad de convenios internos, un programa académico de postgrado adscrito a otras sedes de la Universidad.
\end{itemize}

\textbf{Metadatos}

La construcción de las cifras oficiales de estudiantes matriculados de la Sede Orinoquía, las definiciones que hacen parte de estas así como las codificaciones y clasificaciones aquí empleadas se encuentran contenidas en la sección \textbf{Matriculados} del capítulo de \emph{Metadatos} de las cifras oficiales generales que hacen parte de la página de \href{http://estadisticas.unal.edu.co/home/}{estadísticas} de la Universidad Nacional de Colombia. Invitamos a los lectores a explorar y conocer estos metadatos los cuales, además de orientar y facilitar el entendimiento de la información acá expuesta, se encuentran disponibles en el siguiente enlace.

\begin{itemize}
\tightlist
\item
  \href{http://estadisticas.unal.edu.co/menu-principal/cifras-generales/metadatos/cifras-generales/}{Metadatos Cifras Oficiales Universidad Nacional de Colombia}
\end{itemize}

\hypertarget{MatPre}{%
\section{Matriculados en pregrado}\label{MatPre}}

A continuación, se presenta la información oficial del total de estudiantes matricualdos en pregrado en la Universidad y que han sido admitidos a través de la \textbf{Sede Orinoquía}. Incluye los estudiantes de pregrado que se encuentran en las tres etapas de formación: \emph{inicial}, \emph{movilidad} y \emph{final}. En específico, se presenta la evolución histórica de los matriculados en pregrado desde diferentes perspectivas: general, etapa de formación, sedes andinas, sexo, grupos de edad, estrato socioeconómico, departamentos y municipios de residencia y programas académicos en los que estos se encuentran matriculados. Para cada una de las variables analizadas se presenta la evolución histórica (\emph{serie de tiempo}) así como el comportamiento actual (\emph{estado actual}) derivado de las últimas mediciones disponibles.

\hypertarget{evoluciuxf3n-histuxf3rica-2}{%
\subsection{Evolución histórica}\label{evoluciuxf3n-histuxf3rica-2}}

A continuación, la Figura \ref{fig:F1MatPre}, presenta la evolución histórica -\emph{desde el periodo 20091}-, del total de matriculados en pregrado en la \textbf{Sede Orinoquía}.

\begin{figure}
\centering
\includegraphics{Boletin-Orinoquia_files/figure-latex/F1MatPre-1.pdf}
\caption{\label{fig:F1MatPre}Fuente: Dirección Nacional de Planeación y Estadística con base en información de la Dirección Nacional Información Académica}
\end{figure}

\hypertarget{informaciuxf3n-por-etapa-de-formaciuxf3n}{%
\subsection{Información por etapa de formación}\label{informaciuxf3n-por-etapa-de-formaciuxf3n}}

A continuación, las figuras \ref{fig:F2MatPre} y \ref{fig:F3MatPre} presentan, respectivamente, la evolución histórica y el comportamiento actual del total de matriculados en pregrado en la \textbf{Sede Orinoquía} según la etapa de formación en la que estos se encuentran.

\begin{itemize}
\item
  \emph{Etapa Inicial}: Estudiantes matriculados en pregrado ubicados físicamente en la \emph{Sede Orinoquía}
\item
  \emph{Etapa de movilidad}: Estudiantes matriculados en pregrado ubicados físicamente en las sedes andinas (\emph{Bogotá}, \emph{Medellín}, \emph{Manizales} o \emph{Palmira}).
\end{itemize}

\begin{figure}
\centering
\includegraphics{Boletin-Orinoquia_files/figure-latex/F2MatPre-1.pdf}
\caption{\label{fig:F2MatPre}Fuente: Dirección Nacional de Planeación y Estadística con base en información de la Dirección Nacional Información Académica}
\end{figure}

\begin{figure}
\centering
\includegraphics{Boletin-Orinoquia_files/figure-latex/F3MatPre-1.pdf}
\caption{\label{fig:F3MatPre}Fuente: Dirección Nacional de Planeación y Estadística con base en información de la Dirección Nacional Información Académica}
\end{figure}

\hypertarget{informaciuxf3n-por-sede-andina}{%
\subsection{Información por sede andina}\label{informaciuxf3n-por-sede-andina}}

A continuación, las figuras \ref{fig:F4MatPre} y \ref{fig:F5MatPre} presentan, respectivamente, la evolución histórica y el comportamiento actual del total de estudiantes matriculados de pregrado en la \textbf{Sede Orinoquía} según la sede andina en la que estos cursarán (etapa inicial) o se encuentran cursando sus estudios (etapa de movilidad).

\begin{figure}
\centering
\includegraphics{Boletin-Orinoquia_files/figure-latex/F4MatPre-1.pdf}
\caption{\label{fig:F4MatPre}Fuente: Dirección Nacional de Planeación y Estadística con base en información de la Dirección Nacional Información Académica}
\end{figure}

\begin{figure}
\centering
\includegraphics{Boletin-Orinoquia_files/figure-latex/F5MatPre-1.pdf}
\caption{\label{fig:F5MatPre}Fuente: Dirección Nacional de Planeación y Estadística con base en información de la Dirección Nacional Información Académica}
\end{figure}

\hypertarget{informaciuxf3n-por-sexo-2}{%
\subsection{Información por sexo}\label{informaciuxf3n-por-sexo-2}}

A continuación, las figuras \ref{fig:F6MatPre} y \ref{fig:F7MatPre} presentan, respectivamente, la evolución histórica y el comportamiento actual del total de matriculados en pregrado en la \textbf{Sede Orinoquía} según el sexo biológico.

\begin{figure}
\centering
\includegraphics{Boletin-Orinoquia_files/figure-latex/F6MatPre-1.pdf}
\caption{\label{fig:F6MatPre}Fuente: Dirección Nacional de Planeación y Estadística con base en información de la Dirección Nacional Información Académica}
\end{figure}

\begin{figure}
\centering
\includegraphics{Boletin-Orinoquia_files/figure-latex/F7MatPre-1.pdf}
\caption{\label{fig:F7MatPre}Fuente: Dirección Nacional de Planeación y Estadística con base en información de la Dirección Nacional Información Académica}
\end{figure}

\hypertarget{informaciuxf3n-por-edad-2}{%
\subsection{Información por edad}\label{informaciuxf3n-por-edad-2}}

A continuación, las figuras \ref{fig:F8MatPre} y \ref{fig:F9MatPre} presentan, respectivamente, la evolución histórica y el comportamiento actual del total de estudiantes matriculados en pregrado de la \textbf{Sede Orinoquía} según grupos de edad.

\begin{figure}
\centering
\includegraphics{Boletin-Orinoquia_files/figure-latex/F8MatPre-1.pdf}
\caption{\label{fig:F8MatPre}Fuente: Dirección Nacional de Planeación y Estadística con base en información de la Dirección Nacional Información Académica}
\end{figure}

\begin{figure}
\centering
\includegraphics{Boletin-Orinoquia_files/figure-latex/F9MatPre-1.pdf}
\caption{\label{fig:F9MatPre}Fuente: Dirección Nacional de Planeación y Estadística con base en información de la Dirección Nacional Información Académica}
\end{figure}

\hypertarget{informaciuxf3n-por-estrato}{%
\subsection{Información por estrato}\label{informaciuxf3n-por-estrato}}

A continuación, las figuras \ref{fig:F10MatPre} y \ref{fig:F11MatPre} presentan, respectivamente, la evolución histórica y el comportamiento actual del total de estudiantes matriculados en pregrado en la \textbf{Sede Orinoquía} según el estrato socioeconómico.

\begin{figure}
\centering
\includegraphics{Boletin-Orinoquia_files/figure-latex/F10MatPre-1.pdf}
\caption{\label{fig:F10MatPre}Fuente: Dirección Nacional de Planeación y Estadística con base en información de la Dirección Nacional Información Académica}
\end{figure}

\begin{figure}
\centering
\includegraphics{Boletin-Orinoquia_files/figure-latex/F11MatPre-1.pdf}
\caption{\label{fig:F11MatPre}Fuente: Dirección Nacional de Planeación y Estadística con base en información de la Dirección Nacional Información Académica}
\end{figure}

\hypertarget{tablas-cifras-agregadas-2}{%
\subsection{Tablas cifras agregadas}\label{tablas-cifras-agregadas-2}}

A continuación se presentan, a través de tablas, los agregados/consolidados históricos del total de matriculados en pregrado de la \textbf{Sede Orinoquía} por departamentos y municipios de procedencia así como los programas académicos de pregrado que estos cursan en las sedes andinas de la Universidad (Bogotá, Medellín, Manizales y Palmira).

Los interesados en \emph{imprimir}, \emph{copiar} o \emph{descargar} estas cifras, pueden hacerlo a través de las múltiples opciones que se ofrecen en la parte superior izquierda de cada una de las tablas (Copiar, CSV, Excel, PDF e Imprimir). Así mismo, estas tablas permiten filtrar los resultados por aquellas variables de interés.

\hypertarget{departamentos-y-municipios-de-procedencia-1}{%
\subsubsection{Departamentos y municipios de procedencia}\label{departamentos-y-municipios-de-procedencia-1}}

A continuación, la Tabla \ref{fig:F12MatPre} presenta los acumulados \textbf{históricos}, por \textbf{años y semestres}, en la \textbf{Sede Orinoquía}, de los matriculados en pregrado por departamentos y municipios de procedencia.

\begin{figure}
\centering
\includegraphics{Boletin-Orinoquia_files/figure-latex/F12MatPre-1.pdf}
\caption{\label{fig:F12MatPre}Fuente: Dirección Nacional de Planeación y Estadística con base en información de la Dirección Nacional Información Académica}
\end{figure}

\hypertarget{programas-acaduxe9micos-de-pregrado-1}{%
\subsubsection{Programas académicos de pregrado}\label{programas-acaduxe9micos-de-pregrado-1}}

A continuación, la Tabla \ref{fig:F13MatPre} presenta el acumulado histórico de matriculados en pregrado, en la \textbf{Sede Orinoquía}, por años, semestres, sedes andinas, facultades y programas académicos cursados.

\begin{figure}
\centering
\includegraphics{Boletin-Orinoquia_files/figure-latex/F13MatPre-1.pdf}
\caption{\label{fig:F13MatPre}Fuente: Dirección Nacional de Planeación y Estadística con base en información de la Dirección Nacional Información Académica}
\end{figure}

\hypertarget{MatPreIni}{%
\section{Matriculados en pregrado - Etapa Inicial}\label{MatPreIni}}

A continuación, se presenta la información oficial del total de estudiantes matricualdos en pregrado en la Universidad y que han sido admitidos a través de la \textbf{Sede Orinoquía}. Incluye los estudiantes de pregrado que se encuentran en la \textbf{etapa inicial} del proceso de formación; es decir, \textbf{ubicados de manera presencial} en la Sede Orinoquía. En específico, se presenta la evolución histórica de los matriculados en pregrado en etapa inicial desde diferentes perspectivas: general, sedes andinas, sexo, grupos de edad, estrato socioeconómico, departamentos y municipios de residencia y programas académicos de pregrado en los que estos se encuentran matriculados. Para cada una de las variables analizadas se presenta la evolución histórica (\emph{serie de tiempo}) así como el comportamiento actual (\emph{estado actual}) derivado de las últimas mediciones disponibles.

\hypertarget{evoluciuxf3n-histuxf3rica-3}{%
\subsection{Evolución histórica}\label{evoluciuxf3n-histuxf3rica-3}}

A continuación, la Figura \ref{fig:F1MatPreFI}, presenta la evolución histórica -\emph{desde el periodo 20091}-, del total de matriculados en pregrado -etapa inicial- en la \textbf{Sede Orinoquía}.

\begin{figure}
\centering
\includegraphics{Boletin-Orinoquia_files/figure-latex/F1MatPreFI-1.pdf}
\caption{\label{fig:F1MatPreFI}Fuente: Dirección Nacional de Planeación y Estadística con base en información de la Dirección Nacional Información Académica}
\end{figure}

\hypertarget{informaciuxf3n-por-sede-andina-1}{%
\subsection{Información por sede andina}\label{informaciuxf3n-por-sede-andina-1}}

A continuación, las figuras \ref{fig:F2MatPreFI} y \ref{fig:F3MatPreFI} presentan, respectivamente, la evolución histórica y el comportamiento actual del total de estudiantes matriculados de pregrado - etapa inicial- en la \textbf{Sede Orinoquía} según la sede andina en la que estos cursarán sus estudios de pregrado en la etapa de movilidad.

\begin{figure}
\centering
\includegraphics{Boletin-Orinoquia_files/figure-latex/F2MatPreFI-1.pdf}
\caption{\label{fig:F2MatPreFI}Fuente: Dirección Nacional de Planeación y Estadística con base en información de la Dirección Nacional Información Académica}
\end{figure}

\begin{figure}
\centering
\includegraphics{Boletin-Orinoquia_files/figure-latex/F3MatPreFI-1.pdf}
\caption{\label{fig:F3MatPreFI}Fuente: Dirección Nacional de Planeación y Estadística con base en información de la Dirección Nacional Información Académica}
\end{figure}

\hypertarget{informaciuxf3n-por-sexo-3}{%
\subsection{Información por sexo}\label{informaciuxf3n-por-sexo-3}}

A continuación, las figuras \ref{fig:F4MatPreFI} y \ref{fig:F5MatPreFI} presentan, respectivamente, la evolución histórica y el comportamiento actual del total de matriculados en pregrado -etapa inicial- en la \textbf{Sede Orinoquía} según el sexo biológico.

\begin{figure}
\centering
\includegraphics{Boletin-Orinoquia_files/figure-latex/F4MatPreFI-1.pdf}
\caption{\label{fig:F4MatPreFI}Fuente: Dirección Nacional de Planeación y Estadística con base en información de la Dirección Nacional Información Académica}
\end{figure}

\begin{figure}
\centering
\includegraphics{Boletin-Orinoquia_files/figure-latex/F5MatPreFI-1.pdf}
\caption{\label{fig:F5MatPreFI}Fuente: Dirección Nacional de Planeación y Estadística con base en información de la Dirección Nacional Información Académica}
\end{figure}

\hypertarget{informaciuxf3n-por-edad-3}{%
\subsection{Información por edad}\label{informaciuxf3n-por-edad-3}}

A continuación, las figuras \ref{fig:F6MatPreFI} y \ref{fig:F7MatPreFI} presentan, respectivamente, la evolución histórica y el comportamiento actual del total de estudiantes matriculados en pregrado -etapa inicial- de la \textbf{Sede Orinoquía} según grupos de edad.

\begin{figure}
\centering
\includegraphics{Boletin-Orinoquia_files/figure-latex/F6MatPreFI-1.pdf}
\caption{\label{fig:F6MatPreFI}Fuente: Dirección Nacional de Planeación y Estadística con base en información de la Dirección Nacional Información Académica}
\end{figure}

\begin{figure}
\centering
\includegraphics{Boletin-Orinoquia_files/figure-latex/F7MatPreFI-1.pdf}
\caption{\label{fig:F7MatPreFI}Fuente: Dirección Nacional de Planeación y Estadística con base en información de la Dirección Nacional Información Académica}
\end{figure}

\hypertarget{informaciuxf3n-por-estrato-1}{%
\subsection{Información por estrato}\label{informaciuxf3n-por-estrato-1}}

A continuación, las figuras \ref{fig:F8MatPreFI} y \ref{fig:F9MatPreFI} presentan, respectivamente, la evolución histórica y el comportamiento actual del total de estudiantes matriculados en pregrado -etapa inicial- en la \textbf{Sede Orinoquía} según el estrato socioeconómico.

\begin{figure}
\centering
\includegraphics{Boletin-Orinoquia_files/figure-latex/F8MatPreFI-1.pdf}
\caption{\label{fig:F8MatPreFI}Fuente: Dirección Nacional de Planeación y Estadística con base en información de la Dirección Nacional Información Académica}
\end{figure}

\begin{figure}
\centering
\includegraphics{Boletin-Orinoquia_files/figure-latex/F9MatPreFI-1.pdf}
\caption{\label{fig:F9MatPreFI}Fuente: Dirección Nacional de Planeación y Estadística con base en información de la Dirección Nacional Información Académica}
\end{figure}

\hypertarget{tablas-cifras-agregadas-3}{%
\subsection{Tablas cifras agregadas}\label{tablas-cifras-agregadas-3}}

A continuación se presentan, a través de tablas, los agregados/consolidados históricos del total de matriculados en pregrado -etapa inicial- de la \textbf{Sede Orinoquía} por departamentos y municipios de procedencia así como los programas académicos de pregrado que estos cursan en las sedes andinas de la Universidad (Bogotá, Medellín, Manizales o Palmira).

Los interesados en \emph{imprimir}, \emph{copiar} o \emph{descargar} estas cifras, pueden hacerlo a través de las múltiples opciones que se ofrecen en la parte superior izquierda de cada una de las tablas (Copiar, CSV, Excel, PDF e Imprimir). Así mismo, estas tablas permiten filtrar los resultados por aquellas variables de interés.

\hypertarget{departamentos-y-municipios-de-procedencia-2}{%
\subsubsection{Departamentos y municipios de procedencia}\label{departamentos-y-municipios-de-procedencia-2}}

A continuación, la Tabla \ref{fig:F10MatPreFI} presenta los acumulados \textbf{históricos}, por \textbf{años y semestres}, en la \textbf{Sede Orinoquía}, de los matriculados en pregrado -etapa inicial- por departamentos y municipios de procedencia.

\begin{figure}
\centering
\includegraphics{Boletin-Orinoquia_files/figure-latex/F10MatPreFI-1.pdf}
\caption{\label{fig:F10MatPreFI}Fuente: Dirección Nacional de Planeación y Estadística con base en información de la Dirección Nacional Información Académica}
\end{figure}

\hypertarget{programas-acaduxe9micos-de-pregrado-2}{%
\subsubsection{Programas académicos de pregrado}\label{programas-acaduxe9micos-de-pregrado-2}}

A continuación, la Tabla \ref{fig:F11MatPreFI} presenta el acumulado histórico de matriculados en pregrado -etapa inicial-, en la \textbf{Sede Orinoquía}, por años, semestres, sedes andinas, facultades y programas académicos cursados.

\begin{figure}
\centering
\includegraphics{Boletin-Orinoquia_files/figure-latex/F11MatPreFI-1.pdf}
\caption{\label{fig:F11MatPreFI}Fuente: Dirección Nacional de Planeación y Estadística con base en información de la Dirección Nacional Información Académica}
\end{figure}

\hypertarget{MatPreMov}{%
\section{Matriculados en pregrado - Etapa Movilidad}\label{MatPreMov}}

A continuación, se presenta la información oficial del total de estudiantes matricualdos en pregrado en la Universidad y que han sido admitidos a través de la \textbf{Sede Orinoquía}. Incluye los estudiantes de pregrado que se encuentran en la \textbf{etapa de movilidad} del proceso de formación; es decir, \textbf{ubicados de manera presencial} en las sedes andinas de la Universidad (Bogotá, Medellín, Manizales o Palmira). En específico, se presenta la evolución histórica de los matriculados en pregrado en etapa de movilidad desde diferentes perspectivas: general, sedes andinas de ubicación, sexo, grupos de edad, estrato socioeconómico, departamentos y municipios de residencia y programas académicos de pregrado en los que estos se encuentran matriculados. Para cada una de las variables analizadas se presenta la evolución histórica (\emph{serie de tiempo}) así como el comportamiento actual (\emph{estado actual}) derivado de las últimas mediciones disponibles.

\hypertarget{evoluciuxf3n-histuxf3rica-4}{%
\subsection{Evolución histórica}\label{evoluciuxf3n-histuxf3rica-4}}

A continuación, la Figura \ref{fig:F1MatPreFM}, presenta la evolución histórica -\emph{desde el periodo 20092}-, del total de matriculados en pregrado -etapa de movilidad- en la \textbf{Sede Orinoquía}.

\begin{figure}
\centering
\includegraphics{Boletin-Orinoquia_files/figure-latex/F1MatPreFM-1.pdf}
\caption{\label{fig:F1MatPreFM}Fuente: Dirección Nacional de Planeación y Estadística con base en información de la Dirección Nacional Información Académica}
\end{figure}

\hypertarget{informaciuxf3n-por-sede-andina-2}{%
\subsection{Información por sede andina}\label{informaciuxf3n-por-sede-andina-2}}

A continuación, las figuras \ref{fig:F2MatPreFM} y \ref{fig:F3MatPreFM} presentan, respectivamente, la evolución histórica y el comportamiento actual del total de estudiantes matriculados de pregrado - etapa de movilidad- en la \textbf{Sede Orinoquía} según la sede andina en la que estos se encuentran cursando sus estudios de pregrado en la etapa de movilidad.

\begin{figure}
\centering
\includegraphics{Boletin-Orinoquia_files/figure-latex/F2MatPreFM-1.pdf}
\caption{\label{fig:F2MatPreFM}Fuente: Dirección Nacional de Planeación y Estadística con base en información de la Dirección Nacional Información Académica}
\end{figure}

\begin{figure}
\centering
\includegraphics{Boletin-Orinoquia_files/figure-latex/F3MatPreFM-1.pdf}
\caption{\label{fig:F3MatPreFM}Fuente: Dirección Nacional de Planeación y Estadística con base en información de la Dirección Nacional Información Académica}
\end{figure}

\hypertarget{informaciuxf3n-por-sexo-4}{%
\subsection{Información por sexo}\label{informaciuxf3n-por-sexo-4}}

A continuación, las figuras \ref{fig:F4MatPreFM} y \ref{fig:F5MatPreFM} presentan, respectivamente, la evolución histórica y el comportamiento actual del total de matriculados en pregrado -etapa de movilidad- en la \textbf{Sede Orinoquía} según el sexo biológico.

\begin{figure}
\centering
\includegraphics{Boletin-Orinoquia_files/figure-latex/F4MatPreFM-1.pdf}
\caption{\label{fig:F4MatPreFM}Fuente: Dirección Nacional de Planeación y Estadística con base en información de la Dirección Nacional Información Académica}
\end{figure}

\begin{figure}
\centering
\includegraphics{Boletin-Orinoquia_files/figure-latex/F5MatPreFM-1.pdf}
\caption{\label{fig:F5MatPreFM}Fuente: Dirección Nacional de Planeación y Estadística con base en información de la Dirección Nacional Información Académica}
\end{figure}

\hypertarget{informaciuxf3n-por-edad-4}{%
\subsection{Información por edad}\label{informaciuxf3n-por-edad-4}}

A continuación, las figuras \ref{fig:F6MatPreFM} y \ref{fig:F7MatPreFM} presentan, respectivamente, la evolución histórica y el comportamiento actual del total de estudiantes matriculados en pregrado -etapa de movilidad- de la \textbf{Sede Orinoquía} según grupos de edad.

\begin{figure}
\centering
\includegraphics{Boletin-Orinoquia_files/figure-latex/F6MatPreFM-1.pdf}
\caption{\label{fig:F6MatPreFM}Fuente: Dirección Nacional de Planeación y Estadística con base en información de la Dirección Nacional Información Académica}
\end{figure}

\begin{figure}
\centering
\includegraphics{Boletin-Orinoquia_files/figure-latex/F7MatPreFM-1.pdf}
\caption{\label{fig:F7MatPreFM}Fuente: Dirección Nacional de Planeación y Estadística con base en información de la Dirección Nacional Información Académica}
\end{figure}

\hypertarget{informaciuxf3n-por-estrato-2}{%
\subsection{Información por estrato}\label{informaciuxf3n-por-estrato-2}}

A continuación, las figuras \ref{fig:F8MatPreFM} y \ref{fig:F9MatPreFM} presentan, respectivamente, la evolución histórica y el comportamiento actual del total de estudiantes matriculados en pregrado -etapa de movilidad- en la \textbf{Sede Orinoquía} según el estrato socioeconómico.

\begin{figure}
\centering
\includegraphics{Boletin-Orinoquia_files/figure-latex/F8MatPreFM-1.pdf}
\caption{\label{fig:F8MatPreFM}Fuente: Dirección Nacional de Planeación y Estadística con base en información de la Dirección Nacional Información Académica}
\end{figure}

\begin{figure}
\centering
\includegraphics{Boletin-Orinoquia_files/figure-latex/F9MatPreFM-1.pdf}
\caption{\label{fig:F9MatPreFM}Fuente: Dirección Nacional de Planeación y Estadística con base en información de la Dirección Nacional Información Académica}
\end{figure}

\hypertarget{tablas-cifras-agregadas-4}{%
\subsection{Tablas cifras agregadas}\label{tablas-cifras-agregadas-4}}

A continuación se presentan, a través de tablas, los agregados/consolidados históricos del total de matriculados en pregrado -etapa de movilidad- de la \textbf{Sede Orinoquía} por departamentos y municipios de procedencia así como los programas académicos de pregrado que estos cursan en las sedes andinas de la Universidad (Bogotá, Medellín, Manizales y Palmira).

Los interesados en \emph{imprimir}, \emph{copiar} o \emph{descargar} estas cifras, pueden hacerlo a través de las múltiples opciones que se ofrecen en la parte superior izquierda de cada una de las tablas (Copiar, CSV, Excel, PDF e Imprimir). Así mismo, estas tablas permiten filtrar los resultados por aquellas variables de interés.

\hypertarget{departamentos-y-municipios-de-procedencia-3}{%
\subsubsection{Departamentos y municipios de procedencia}\label{departamentos-y-municipios-de-procedencia-3}}

A continuación, la Tabla \ref{fig:F10MatPreFM} presenta los acumulados \textbf{históricos}, por \textbf{años y semestres}, en la \textbf{Sede Orinoquía}, de los matriculados en pregrado -etapa de movilidad- por departamentos y municipios de procedencia.

\begin{figure}
\centering
\includegraphics{Boletin-Orinoquia_files/figure-latex/F10MatPreFM-1.pdf}
\caption{\label{fig:F10MatPreFM}Fuente: Dirección Nacional de Planeación y Estadística con base en información de la Dirección Nacional Información Académica}
\end{figure}

\hypertarget{programas-acaduxe9micos-de-pregrado-3}{%
\subsubsection{Programas académicos de pregrado}\label{programas-acaduxe9micos-de-pregrado-3}}

A continuación, la Tabla \ref{fig:F11MatPreFM} presenta el acumulado histórico de matriculados en pregrado -etapa de movilidad-, en la \textbf{Sede Orinoquía}, por años, semestres, sedes andinas, facultades y programas académicos cursados.

\begin{figure}
\centering
\includegraphics{Boletin-Orinoquia_files/figure-latex/F11MatPreFM-1.pdf}
\caption{\label{fig:F11MatPreFM}Fuente: Dirección Nacional de Planeación y Estadística con base en información de la Dirección Nacional Información Académica}
\end{figure}

\hypertarget{Grad}{%
\chapter{Graduados}\label{Grad}}

Este capítulo presenta el consolidado de las principales características asociadas a la información estadística oficial de los estudiantes graduados en pregrado y postgrado en la sede Orinoquía de la Universidad Nacional de Colombia.

El proceso de formación en pregrado en la Sede Orinoquía se da a la luz de los lineamientos definidos en el Programa Especial de Admisión y Movilidad Académica \protect\hyperlink{peama}{PEAMA}. Los aspirantes admitidos a pregrado en la Sede Orinoquía cursan la fase inicial de su proceso de formación en la Sede y posteriormente realizan la fase de movilidad en una de las sedes andinas de la Universidad Nacional de Colombia (Bogotá, Medellín, Manizales o Plamira) en donde, además de cursar buena parte de su proceso de formación, reciben su título de pregrado. Los estudiantes graduados en pregrado de la Sede Orinoquía y que hacen parte del presente capítulo, por lo antes expuesto, corresponde a aquellos que ingresaron a esta Sede a través del programa \protect\hyperlink{peama}{PEAMA} y que posteriormente se graduaron en una de las sedes andinas de la Universidad. En contraste, la formación postgradual en la Sede Orinoquía se da a través de programas propios; es decir, los estudiantes graduados en este nivel de formación cursaron la totalidad de sus estudios en la Sede.

A continuación, se presenta una breve descripción de las secciones que hacen parte de este capítulo así como la ubicación del sitio web en donde se presentan las definiciones, los estándares y las codificaciones/clasificaciones que hacen parte de la información acá contenida (\emph{metadatos}) y cuya exploración y lectura, sin duda, facilitará el entendimiento de las cifras asociadas a las poblaciones de graduados en esta sede de la Universidad.

\textbf{Secciones}

\begin{itemize}
\item
  \protect\hyperlink{GraPre}{Graduados en pregrado}: contiene la información oficial del total de graduados en pregrado en la Sede Orinoquía de la Universidad Nacional de Colombia y que se incribieron, fueron admitidos y cursaron su proceso de formación a través del Programa Especial de Admisión y Movilidad Académica \protect\hyperlink{peama}{PEAMA}.
\item
  \protect\hyperlink{GraPos}{Graduados en postgrado}: contiene la información oficial del total de graduados en los programas de postgrado adscritos a la Sede Orinoquía de la Universidad Nacional de Colombia.
\end{itemize}

\textbf{Metadatos}

La construcción de las cifras oficiales de graduados en la Sede Orinoquía, las definiciones que hacen parte de estas así como las codificaciones y clasificaciones aquí empleadas se encuentran contenidas en la sección \textbf{Graduados} del capítulo de \emph{Metadatos} de las cifras oficiales generales que hacen parte de la página de \href{http://estadisticas.unal.edu.co/home/}{estadísticas} de la Universidad Nacional de Colombia. Invitamos a los lectores a explorar y conocer estos metadatos los cuales, además de orientar y facilitar el entendimiento de la información acá expuesta, se encuentran disponibles en el siguiente enlace.

\begin{itemize}
\tightlist
\item
  \href{http://estadisticas.unal.edu.co/menu-principal/cifras-generales/metadatos/cifras-generales/}{Metadatos Cifras Oficiales Universidad Nacional de Colombia}
\end{itemize}

\hypertarget{GraPre}{%
\section{Graduados en pregrado}\label{GraPre}}

A continuación, se presentan las principales características asociadas a los estudiantes graduados en pregrado de la \textbf{Sede Orinoquía} de la Universidad Nacional de Colombia. En específico, se presenta la evolución histórica de los graduados de pregrado desde diferentes perspectivas: general, sede andina de graduación, sexo, grupos de edad, estrato socioeconómico, departamentos y municipios de nacimiento y características de la sede de graduación de estos (sede andina, facultad y programas académicos). Para cada una de las variables analizadas se presenta la evolución histórica (\emph{serie de tiempo}) así como el comportamiento actual (\emph{estado actual}) derivado de las últimas mediciones disponibles.

\hypertarget{evoluciuxf3n-histuxf3rica-5}{%
\subsection{Evolución Histórica}\label{evoluciuxf3n-histuxf3rica-5}}

A continuación, la Figura \ref{fig:F1GraPre}, presenta la evolución histórica -\emph{desde el periodo 20131}-, del total de graduados de pregrado en la Sede.

\includegraphics{Boletin-Orinoquia_files/figure-latex/F1GraPre-1.pdf}

\hypertarget{informaciuxf3n-por-sede-andina-3}{%
\subsection{Información por sede andina}\label{informaciuxf3n-por-sede-andina-3}}

A continuación, las figuras \ref{fig:F2GraPre} y \ref{fig:F3GraPre} presentan, respectivamente, la evolución histórica y el comportamiento actual del total de graduados en pregrado según la sede andina de graduación (Bogotá, Medellín, Manizales o palmira).

\includegraphics{Boletin-Orinoquia_files/figure-latex/F2GraPre-1.pdf}
\includegraphics{Boletin-Orinoquia_files/figure-latex/F3GraPre-1.pdf}

\hypertarget{informaciuxf3n-por-sexo-5}{%
\subsection{Información por sexo}\label{informaciuxf3n-por-sexo-5}}

A continuación, las figuras \ref{fig:F4GraPre} y \ref{fig:F5GraPre} presentan, respectivamente, la evolución histórica y el comportamiento actual del total de graduados en pregrado según el sexo biológico.

\includegraphics{Boletin-Orinoquia_files/figure-latex/F4GraPre-1.pdf}
\includegraphics{Boletin-Orinoquia_files/figure-latex/F5GraPre-1.pdf}

\hypertarget{informaciuxf3n-por-edad-5}{%
\subsection{Información por edad}\label{informaciuxf3n-por-edad-5}}

A continuación, las figuras \ref{fig:F6GraPre} y \ref{fig:F7GraPre} presentan, respectivamente, la evolución histórica y el comportamiento actual del total de graduados en pregrado según grupos de edad.

\includegraphics{Boletin-Orinoquia_files/figure-latex/F6GraPre-1.pdf}
\includegraphics{Boletin-Orinoquia_files/figure-latex/F7GraPre-1.pdf}

\hypertarget{informaciuxf3n-por-estrato-3}{%
\subsection{Información por estrato}\label{informaciuxf3n-por-estrato-3}}

A continuación, las figuras \ref{fig:F8GraPre} y \ref{fig:F9GraPre} presentan, respectivamente, la evolución histórica y el comportamiento actual del total de graduados en pregrado según el estrato socioeconómico.

\includegraphics{Boletin-Orinoquia_files/figure-latex/F8GraPre-1.pdf}
\includegraphics{Boletin-Orinoquia_files/figure-latex/F9GraPre-1.pdf}

\hypertarget{tablas-cifras-agregadas-5}{%
\subsection{Tablas cifras agregadas}\label{tablas-cifras-agregadas-5}}

A continuación se presentan, a través de tablas, los agregados/consolidados históricos del total de graduados en pregrado de la \textbf{Sede Orinoquía} por departamentos y municipios de nacimiento así como las sedes andinas (Bogotá, Medellín, Manizales y Palmira), las facultades y los programas académicos de los que estos egresaron.

Los interesados en \emph{imprimir}, \emph{copiar} o \emph{descargar} estas cifras, pueden hacerlo a través de las múltiples opciones que se ofrecen en la parte superior izquierda de cada una de las tablas (Copiar, CSV, Excel, PDF e Imprimir). Así mismo, estas tablas permiten filtrar los resultados por aquellas variables de interés.

\hypertarget{departamentos-y-municipios-de-nacimiento}{%
\subsubsection{Departamentos y municipios de nacimiento}\label{departamentos-y-municipios-de-nacimiento}}

A continuación, la Tabla \ref{fig:F10GraPre} presenta los acumulados \textbf{históricos}, por \textbf{años y semestres}, en la \textbf{Sede Orinoquía}, de los estudiantes graduados en pregrado por departamentos y municipios de nacimiento.

\begin{figure}
\centering
\includegraphics{Boletin-Orinoquia_files/figure-latex/F10GraPre-1.pdf}
\caption{\label{fig:F10GraPre}Fuente: Dirección Nacional de Planeación y Estadística con base en información de la Secretaría General}
\end{figure}

\hypertarget{sedes-andinas-y-programas-acaduxe9micos-de-graduaciuxf3n}{%
\subsubsection{Sedes andinas y programas académicos de graduación}\label{sedes-andinas-y-programas-acaduxe9micos-de-graduaciuxf3n}}

A continuación, la Tabla \ref{fig:F11GraPre} presenta los acumulados \textbf{históricos}, por \textbf{años y semestres}, en la \textbf{Sede Orinoquía}, de los estudiantes graduados en pregrado por sedes andinas, facultades y programas académicos de graduación.

\begin{figure}
\centering
\includegraphics{Boletin-Orinoquia_files/figure-latex/F11GraPre-1.pdf}
\caption{\label{fig:F11GraPre}Fuente: Dirección Nacional de Planeación y Estadística con base en información de la Secretaría General}
\end{figure}

\hypertarget{Doc}{%
\chapter{Docentes}\label{Doc}}

Este capítulo presenta el consolidado de las principales características asociadas a la información estadística oficial de los docentes de carrera adscritos a la Sede Orinoquía de la Universidad Nacional de Colombia así como la información estadística del personal que apoya las labores académicas en esta sede y que se encuentran vinculados a través de otras formas de contratación.

A continuación, se presenta una breve descripción de las secciones que hacen parte de este capítulo así como la ubicación del sitio web en donde se presentan las definiciones, los estándares y las codificaciones/clasificaciones que hacen parte de la información acá contenida (\emph{metadatos}) y cuya exploración y lectura, sin duda, facilitará el entendimiento de las cifras asociadas, principalmente, a la población de docentes de carrera en esta sede de la Universidad.

\textbf{Secciones}

\begin{itemize}
\item
  \protect\hyperlink{DocCar}{Docentes de carrera}: contiene la información oficial del total de docentes de carrera adscritos a la Sede Orinoquía de la Universidad Nacional de Colombia y que fueron vinculados mediante concurso profesoral abierto y público o por reingreso a la carrera docente.
\item
  \protect\hyperlink{DocPEAMA}{Docentes PEAMA}: contiene la información oficial del total de \ldots..
\item
  \protect\hyperlink{DocApo}{Profesionales de apoyo PEAMA}: contiene la información oficial del total de \ldots..
\end{itemize}

\textbf{Metadatos}

La construcción de las cifras oficiales de docentes de carrera en la Sede Orinoquía, las definiciones que hacen parte de esta categoría así como las codificaciones y clasificaciones aquí empleadas se encuentran contenidas en la sección \textbf{Docentes} del capítulo de \emph{Metadatos} de las cifras oficiales generales que hacen parte de la página de \href{http://estadisticas.unal.edu.co/home/}{estadísticas} de la Universidad Nacional de Colombia. Invitamos a los lectores a explorar y conocer estos metadatos los cuales, además de orientar y facilitar el entendimiento de la información de docentes de carrera, se encuentran disponibles en el siguiente enlace.

\begin{itemize}
\tightlist
\item
  \href{http://estadisticas.unal.edu.co/menu-principal/cifras-generales/metadatos/cifras-generales/}{Metadatos Cifras Oficiales Universidad Nacional de Colombia}
\end{itemize}

\hypertarget{DocCar}{%
\section{Docentes de carrera}\label{DocCar}}

A continuación, se presentan las principales características asociadas a los docentes de carrera de la \textbf{Sede Orinoquía} de la Universidad Nacional de Colombia. En específico, se presenta la evolución histórica de los docentes desde diferentes perspectivas: general, sexo, grupos de edad y máximo nivel de formación. Para cada una de las variables analizadas se presenta la evolución histórica (\emph{serie de tiempo}) así como el comportamiento actual (\emph{estado actual}) derivado de las últimas mediciones disponibles.

Nota: Número de docentes de carrera

El número de docentes de carrera en la Sede Orinoquía de la Universidad es bajo. Este hecho implica leer con precaución las frecuencias relativas (porcentajes) presentes en algunos de los resultados que se presentan a continuación. Estos, por por los bajos tamaños poblacionales existentes, son altamentente inestables/volátiles.

\hypertarget{evoluciuxf3n-histuxf3rica-6}{%
\subsection{Evolución Histórica}\label{evoluciuxf3n-histuxf3rica-6}}

A continuación, la Figura \ref{fig:F1DocCar}, presenta la evolución histórica -\emph{desde el periodo 20082}-, del total de docentes de carrera de la \textbf{Sede Orinoquía}.

\includegraphics{Boletin-Orinoquia_files/figure-latex/F1DocCar-1.pdf}

\hypertarget{informaciuxf3n-por-sexo-6}{%
\subsection{Información por sexo}\label{informaciuxf3n-por-sexo-6}}

A continuación, las figuras \ref{fig:F2DocCar} y \ref{fig:F3DocCar} presentan, respectivamente, la evolución histórica y el comportamiento actual del total de docentes de carrera de la \textbf{Sede Orinoquía} según el sexo biológico.

\begin{figure}
\centering
\includegraphics{Boletin-Orinoquia_files/figure-latex/F2DocCar-1.pdf}
\caption{\label{fig:F2DocCar}Fuente: Dirección Nacional de Planeación y Estadística con base en información de la Dirección Nacional de Talento Humano}
\end{figure}

\includegraphics{Boletin-Orinoquia_files/figure-latex/F3DocCar-1.pdf}

\hypertarget{informaciuxf3n-por-edad-6}{%
\subsection{Información por edad}\label{informaciuxf3n-por-edad-6}}

A continuación, las figuras \ref{fig:F4DocCar} y \ref{fig:F5DocCar} presentan, respectivamente, la evolución histórica y el comportamiento actual del total de docentes de carrera de la \textbf{Sede Orinoquía} según grupos de edad.

\begin{figure}
\centering
\includegraphics{Boletin-Orinoquia_files/figure-latex/F4DocCar-1.pdf}
\caption{\label{fig:F4DocCar}Fuente: Dirección Nacional de Planeación y Estadística con base en información de la Dirección Nacional de Talento Humano}
\end{figure}

\includegraphics{Boletin-Orinoquia_files/figure-latex/F5DocCar-1.pdf}

\hypertarget{informaciuxf3n-por-muxe1ximo-nivel-de-formaciuxf3n}{%
\subsection{Información por máximo nivel de formación}\label{informaciuxf3n-por-muxe1ximo-nivel-de-formaciuxf3n}}

A continuación, las figuras \ref{fig:F6DocCar} y \ref{fig:F7DocCar} presentan, respectivamente, la evolución histórica y el comportamiento actual del total de docentes de carrera de la \textbf{Sede Orinoquía} según el máximo nivel de formación.

\begin{figure}
\centering
\includegraphics{Boletin-Orinoquia_files/figure-latex/F6DocCar-1.pdf}
\caption{\label{fig:F6DocCar}Fuente: Dirección Nacional de Planeación y Estadística con base en información de la Dirección Nacional de Talento Humano}
\end{figure}

\begin{figure}
\centering
\includegraphics{Boletin-Orinoquia_files/figure-latex/F7DocCar-1.pdf}
\caption{\label{fig:F7DocCar}Fuente: Dirección Nacional de Planeación y Estadística con base en información de la Dirección Nacional de Talento Humano}
\end{figure}

\hypertarget{DocPEAMA}{%
\section{Docentes PEAMA}\label{DocPEAMA}}

\hypertarget{DocApo}{%
\section{Profesionales de apoyo PEAMA}\label{DocApo}}

\hypertarget{Admi}{%
\chapter{Administrativos}\label{Admi}}

Este capítulo presenta el consolidado de las principales características asociadas a la información estadística oficial de los administrativos de carrera adscritos a la Sede Orinoquía de la Universidad Nacional de Colombia así como la información estadística del personal adicional que apoya las labores administrativas en esta sede y que se encuentran vinculados a través de contratos de prestación de servicios (contratistas).

A continuación, se presenta una breve descripción de las secciones que hacen parte de este capítulo así como la ubicación del sitio web en donde se presentan las definiciones, los estándares y las codificaciones/clasificaciones que hacen parte de la información acá contenida (\emph{metadatos}) y cuya exploración y lectura, sin duda, facilitará el entendimiento de las cifras asociadas, principalmente, a la población de administrativos de carrera en esta sede de la Universidad.

\textbf{Secciones}

\begin{itemize}
\item
  \protect\hyperlink{AdmCar}{Administrativos de carrera}: contiene la información oficial del total de funcionarios administrativos de carrera adscritos a la Sede Orinoquía de la Universidad Nacional de Colombia y que fueron vinculados mediante concurso público.
\item
  \protect\hyperlink{Contrat}{Contratistas}: contiene la información oficial del total de \ldots..
\end{itemize}

\textbf{Metadatos}

La construcción de las cifras oficiales de administrativos de carrera en la Sede Orinoquía, las definiciones que hacen parte de esta categoría así como las codificaciones y clasificaciones aquí empleadas se encuentran contenidas en la sección \textbf{Administrativos} del capítulo de \emph{Metadatos} de las cifras oficiales generales que hacen parte de la página de \href{http://estadisticas.unal.edu.co/home/}{estadísticas} de la Universidad Nacional de Colombia. Invitamos a los lectores a explorar y conocer estos metadatos los cuales, además de orientar y facilitar el entendimiento de la información de administrativos de carrera, se encuentran disponibles en el siguiente enlace.

\begin{itemize}
\tightlist
\item
  \href{http://estadisticas.unal.edu.co/menu-principal/cifras-generales/metadatos/cifras-generales/}{Metadatos Cifras Oficiales Universidad Nacional de Colombia}
\end{itemize}

\hypertarget{AdmCar}{%
\section{Administrativos de carrera}\label{AdmCar}}

A continuación, se presentan las principales características asociadas a los administrativos de carrera de la \textbf{Sede Orinoquía} de la Universidad Nacional de Colombia. En específico, se presenta la evolución histórica de los funcionarios administrativos desde diferentes perspectivas: general, sexo, grupos de edad y máximo nivel de formación. Para cada una de las variables analizadas se presenta la evolución histórica (\emph{serie de tiempo}) así como el comportamiento actual (\emph{estado actual}) derivado de las últimas mediciones disponibles.

Nota: Número de administrativos de carrera

El número de administrativos de carrera en la Sede Orinoquía de la Universidad es bajo. Este hecho implica leer con precaución las frecuencias relativas (porcentajes) presentes en algunos de los resultados que se presentan a continuación. Estos, por por los bajos tamaños poblacionales existentes, son altamentente inestables/volátiles.

\hypertarget{evoluciuxf3n-histuxf3rica-7}{%
\subsection{Evolución Histórica}\label{evoluciuxf3n-histuxf3rica-7}}

A continuación, la Figura \ref{fig:F1AdmiCar}, presenta la evolución histórica -\emph{desde el periodo 20082}-, del total de funcionarios administrativos de carrera de la \textbf{Sede Orinoquía}.

\includegraphics{Boletin-Orinoquia_files/figure-latex/F1AdmiCar-1.pdf}

\hypertarget{informaciuxf3n-por-sexo-7}{%
\subsection{Información por sexo}\label{informaciuxf3n-por-sexo-7}}

A continuación, las figuras \ref{fig:F2AdmiCar} y \ref{fig:F3AdmiCar} presentan, respectivamente, la evolución histórica y el comportamiento actual del total de funcionarios administrativos de carrera de la \textbf{Sede Orinoquía} según el sexo biológico.

\begin{figure}
\centering
\includegraphics{Boletin-Orinoquia_files/figure-latex/F2AdmiCar-1.pdf}
\caption{\label{fig:F2AdmiCar}Fuente: Dirección Nacional de Planeación y Estadística con base en información de la Dirección Nacional de Talento Humano}
\end{figure}

\includegraphics{Boletin-Orinoquia_files/figure-latex/F3AdmiCar-1.pdf}

\hypertarget{informaciuxf3n-por-edad-7}{%
\subsection{Información por edad}\label{informaciuxf3n-por-edad-7}}

A continuación, las figuras \ref{fig:F4AdmiCar} y \ref{fig:F5AdmiCar} presentan, respectivamente, la evolución histórica y el comportamiento actual del total de funcionarios administrativos de carrera de la \textbf{Sede Orinoquía} según grupos de edad.

\begin{figure}
\centering
\includegraphics{Boletin-Orinoquia_files/figure-latex/F4AdmiCar-1.pdf}
\caption{\label{fig:F4AdmiCar}Fuente: Dirección Nacional de Planeación y Estadística con base en información de la Dirección Nacional de Talento Humano}
\end{figure}

\includegraphics{Boletin-Orinoquia_files/figure-latex/F5AdmiCar-1.pdf}

\hypertarget{informaciuxf3n-por-muxe1ximo-nivel-de-formaciuxf3n-1}{%
\subsection{Información por máximo nivel de formación}\label{informaciuxf3n-por-muxe1ximo-nivel-de-formaciuxf3n-1}}

A continuación, las figuras \ref{fig:F6AdmiCar} y \ref{fig:F7AdmiCar} presentan, respectivamente, la evolución histórica y el comportamiento actual del total de funcionarios administrativos de carrera de la \textbf{Sede Orinoquía} según el máximo nivel de formación.

\begin{figure}
\centering
\includegraphics{Boletin-Orinoquia_files/figure-latex/F6AdmiCar-1.pdf}
\caption{\label{fig:F6AdmiCar}Fuente: Dirección Nacional de Planeación y Estadística con base en información de la Dirección Nacional de Talento Humano}
\end{figure}

\begin{figure}
\centering
\includegraphics{Boletin-Orinoquia_files/figure-latex/F7AdmiCar-1.pdf}
\caption{\label{fig:F7AdmiCar}Fuente: Dirección Nacional de Planeación y Estadística con base en información de la Dirección Nacional de Talento Humano}
\end{figure}

\hypertarget{Contrat}{%
\section{Contratistas}\label{Contrat}}

\hypertarget{Inv}{%
\chapter{Investigación, Extensión e Innovación}\label{Inv}}

\hypertarget{investigaciuxf3n}{%
\section{Investigación}\label{investigaciuxf3n}}

\hypertarget{extensiuxf3n}{%
\section{Extensión}\label{extensiuxf3n}}

\hypertarget{Bie}{%
\chapter{Bienestar}\label{Bie}}

\hypertarget{gestiuxf3n-y-fomento-socioeconuxf3mico}{%
\section{Gestión y fomento socioeconómico}\label{gestiuxf3n-y-fomento-socioeconuxf3mico}}

  \bibliography{book.bib,packages.bib}

\end{document}
